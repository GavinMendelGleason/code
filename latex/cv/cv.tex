\documentclass[a4paper,11pt]{moderncv}

\usepackage[british]{babel}
\usepackage[T1]{fontenc}
\usepackage[utf8]{inputenc}
\usepackage{lmodern}
\usepackage{microtype}
%\usepackage{amsfonts}
%\usepackage{slantsc}
%\usepackage{microtype}
\usepackage{textcase}

\newcommand{\cvreference}[7]{%
  \textbf{#1}\newline% Name
  \ifthenelse{\equal{#2}{}}{}{\addresssymbol~#2\newline}%
  \ifthenelse{\equal{#3}{}}{}{#3\newline}%
  \ifthenelse{\equal{#4}{}}{}{#4\newline}%
  \ifthenelse{\equal{#5}{}}{}{#5\newline}%
  \ifthenelse{\equal{#6}{}}{}{\emailsymbol~\texttt{#6}\newline}%
  \ifthenelse{\equal{#7}{}}{}{\phonesymbol~#7}}

%\usepackage{hyperref}

\moderncvtheme{casual}
%\moderncvtheme{fancy}
%\moderncvtheme{banking}
%\moderncvtheme{classic}

%\setlength{\hintscolumnwidth}{45mm}
%\AtBeginDocument{\recomputelengths}

\firstname{Dr. Gavin}
\familyname{Mendel-Gleason}
\title{Research Fellow}
\address{169 St. Attracta Road}{Cabra, Dublin 7}
\mobile{085 136 8737}
%\phone{01 02 03 04 05}
%\fax{01 02 03 04 05}
\email{jacobian@gmail.com}
%\extrainfo{Marié \\ beaucoup d'enfants}
\photo{side-angle-single.png}

%\quote{}

\begin{document}

\maketitle

\section{Employment}

\cventry{2015 -- 2018}{Research Fellow}{Trinity College Dublin}{Dublin}{}
{
  \begin{itemize}
  \item Co-lead of development with Kevin Feeney on the Dacura project, a semantic web ontology driven platform for data-storage and data-entry.
  \item Developed algorithms and tools for proving properties of RDF graphs based on OWL ontology information.
  \item Launched the Seshat Global Historical Databank (http://dacura.scss.tcd.ie/seshat/index.html) using Dacura technology.
  \end{itemize}
}

\cventry{2012 -- 2015}{Postdoctoral Researcher}{Dublin City University}{Dublin}{}
{
  \begin{itemize}
  \item A post-doctoral position with a team of four developers including the principle investigator (Deirdre Hogan).
  \item Lead developer for the QuAS question answering software. This project developed an information retrieval based question answering system which leveraged customer support interactions. It is a cloud based project with a web and mobile interface.
  \item Lead developer for the Atabot project, a platform for using machine learning techniques to recognise and classify user generated content.
  \end{itemize}
}

\cventry{2005 -- 2007}{Lead Software Engineer}{Cognotent Ltd.}{Dublin}{}
{
  \begin{itemize}
  \item Lead developer for irishpressreleases.ie, a CMS for press releases from companies or their PR firms. I wrote a probabilistic parser for automatic data entry. The implementation of the probabilistic parser was written in Common Lisp.
  \item Lead developer for irishblogs.ie. I implemented vector based similarity measures between blog posts using a modified k-means and a longest common sub-sequence algorithm based on Patricia tries to extract news story titles. I architected the treatment of time evolved clustring of news stories. 
  \item All of the research, mathematical development, architecture and much of the implementation was done by myself.
  \end{itemize}
}

\cventry{2004 -- 2005}{Chief Technology Officer}{Aeroglyph Inc.}{Marion, Montana}{}
{
  \begin{itemize}
  \item I worked primarily as an architect and programmer of a logic database system with a semantics similar to a persistent Prolog utilising a triple-store. Elements of our design included a novel mechanism similar to tabling (as used by David Warren in XSB) using continuations, unfold/fold query optimisation, and a novel indexing data structure for persistently storing triples.
  \item I wrote most of the low level programming (memory pager and inner loops) in C.
  \item We successfully completed the development cycle and delivered one server to a client who is currently using it for storing pathology information (tens of thousands of records). Unfortunately funding ran out and the project was discontinued.
  \end{itemize}
}

\cventry{2003 -- 2004}{Lecturer}{Carnegie Mellon University}{Pittsburgh, Pennsylvania}{}
{
  \begin{itemize}
  \item Lecturer and teaching assistant. I was responsible for teaching the problem sessions for engineering physics.
  \end{itemize}
}

\cventry{2003}{Research Assistant}{University of New Mexico}{Albuquerque, New Mexico}{}
{
  \begin{itemize}
  \item Developed finite difference codes for simulation of complex non-linear differential equations for application in biological physics. Code was written in MatLab and Lisp.
  \end{itemize}
}

\cventry{2000-2002}{Software Engineer}{MDLi}{San Francisco, California}{}
{
  \begin{itemize}
  \item Helped write and maintain combinatorial chemistry informatics software for the leading reaction based software product, Afferent.
  \item I participated in development through the end of one release cycle (3.0) and entire release cycle (3.1, a major release) and the beginning of a third (4.0) before the project was canceled due to changes in management. At this point I returned to school to complete my degree.
  \item In charge of automation (robotics) and bug-fixes in the reaction based chemistry engine on Afferent 3.0.
  \item In 3.1 I substantially re-factored the searching mechanism (object-relational mapping) to be thread safe for performance reasons. I designed and implemented the object serialization system for a very complex object system written in CLOS on Afferent 4.0.
  \end{itemize}
}

\section{References}

\subsection{Ireland}

\cvreference{Dr. Declan O'Sullivan}{Professor}{}{School of Computer Science and Statistics}{Trinity College Dublin}{declan.osullivan@scss.tcd.ie}{}

%\cvreference{Dr. Kevin Feeney}{Senior Research Fellow}{School of Computer Science and Statistics}{Trinity College Dublin}{Dublin}{kevin.feeney@cs.tcd.ie}{}

\cvreference{Dr. Deirdre Hogan}{Staff Applied Scientist}{}{LinkedIn}{Dublin}{deirdre.hogan@gmail.com}{}
  
\cvreference{Roger Galligan}{CEO}{}{Cognotent Ltd.}{Dublin}{roger.galligan@cognotent.com}{}

\subsection{United States}

\cvreference{Dr. Randy Gobbel}{Alumnus}{}{SRI International}{San Francisco}{gobbel@acm.org}{}



\section{Personal Experience}

\cvlistitem{Expert in many programming languages including C, C++, Java, Php, Lisp, Haskell,
  OCaML, Prolog, SQL}
\cvlistitem{Expert knowledge of semantic web foundations including semantics of RDF and OWL}
\cvlistitem{Experience in model driven approaches to development}
\cvlistitem{Extensive knowledge of logic, mathematics, statistics and statistical methods}
\cvlistitem{I have used machine learning techniques for text classification and clustering}
\cvlistitem{Implementation experience with program transformation and optimisation techniques for functional, logcial and relational languages}
\cvlistitem{Experience working with several proof assistants, including Coq and Agda}
\cvlistitem{I have worked with ``big data'' in an information retrieval, machine learning and data mining setting}
\cvlistitem{Experience using and customising the Xapian and Lucene search engines}


%\cvcomputer{Humour}{Les histoires de mon oncle Anatole.}
%           {Enfumage}{Je parle beaucoup...}
%\cvcomputer{Opportunisme}{Toutes les occasions sont bonnes àprendre.}{}{}

\section{Education}

\cventry{2007 -- 2011}{PhD Computer Science}
{Dublin City University}{Dublin}{}
{
  In 2007 I began a funded PhD program at Dublin City University in the School of Computing. My
  thesis title is ``Types and Inhabitation of Infinite State Systems''. As part of my PhD I developed a supercompiler written in Haskell and mechanised proofs of correctness in the proof assistant Coq.
}
\cventry{1999 -- 2004}{BS Pure Math, BS Physics}{University of New Mexico}{Albuquerque, New Mexico}{}{
  Graduated Magna Cum Laude with a Bachelor of Science in Pure Math, and a Bachelor of Science in Physics. My GPA was 3.70. I was given the ``Feynman Award'' for excellent academic achievement in modern physics, and was awarded the title of ``Best Student in Pure Math'' by the Math Dept.
}

%\section{Langues étrangères}
%\cvlanguage{Belge}{courant}{langue maternelle}
%\cvlanguage{Espagnol}{notions}{}

%\section{Activités extra-professionnelles}

%\cvline{Association}{Président du « Volant Club »}

%\cvline{Association}{Président de la société philanthropique « Les Joyeux Turlurons »}

\nocite{*}
\bibliographystyle{plain}
\bibliography{publications}

\end{document}
